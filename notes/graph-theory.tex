
\section{Graph basics} 
\label{section:graph-basics} 

\subsection{definitions} 

Gragh $ G = (V, E) $ where $V$ is the set of graph nodes 
and $E$ is the set of edges that connect the nodes. 

\paragraph{Incidence Matrix}
A 0-1 matrix which has V rows and E columns. $(v,e)=1$ iff v is incident upon edge e.
The incidence matrix $\nabla$ of a graph and adjacency matrix A of its {\em line graph} are related by:
$$ A= {\nabla}^{T}\nabla - 2 I. $$

\paragraph{Laplacian Matrix},
also called admittance matrix or Kirchhoff matrix, of a graph $G=(V,E)$ is an undirected, unweighted graph without graph loops $(i, i)$ or multiple edges from one node
to another. Its size is $n \times n$ ($n=|V|$) symmetric matrix defined by
$$ L = D - A $$
where $D = diag(d_1, d_2, \dots, d_n)$ is the degree matrix, and $A$ is the adjacency matrix.

$$ L = {\nabla}^{T}\nabla $$

\subparagraph{normalized version} of the Laplacian matrix is:
$$ L^{norm}(G) =
  \begin{cases}
    1 & \text{if $i=j$ and $d_j \not= 0$} \\
    - \frac{1}{\sqrt{d_i d_j}} & \text{if $i$ and $j$ are adjacent} \\
    0 & \text{otherwise}
  \end{cases}
$$

or:

$$ L^{norm}(G) = I - D^{- \frac{1}{2}} A D^{- \frac{1}{2}} $$

All eigenvalues of the normalized Laplacian are real and non-negative. If $\lambda$ is an eigenvalue of $L$, then $ 0 \le \lambda \le 2 $. These eigenvalues are the spectra of the normalized Laplacian.

\subsection{Spectral Graph Theory}

\paragraph{Spectral Theory}
for symmetric matrix: there exist n mutually orthogonal unit vectors $\psi_1, \psi_2, \ldots, \psi_n $ and numbers $ \lambda_1, \lambda_2, \ldots, \lambda_n $ such that
$\psi_i$ is an eigenvector of eigenvallue $\lambda_i$.

Useful characterization: optimization of {\em Rayleigh quotient}:
$$ \frac{x^T M x}{x^T x} $$
Easy to prove that if $ x = \psi $, its Rayleigh quotient is $ \lambda $.

\begin{theorem}
  The maximum vector that maximizes Rayleigh quotient is an eigenvector associated with the maximum eigenvalue. (Similar for minimum)
  (prove using matrix derivative)
\end{theorem}

\paragraph{Isoperimetry and $ \lambda_2 $ } The {\em isoperimetric ratio} of S, a sub-graph is:
$$ \partial (S) \equiv { (u, v) \in E: u \in S, v \not\in S}. $$
The {\em isoperimetric number} of a graph is the minimum isoperimetric number over all sets of at most half the vertices:
$$ \theta_G \equiv \min_{|S| \le n/2} \theta (S). $$
Lower bound on $ \theta_G $:

\begin{theorem}
  For every $ S \subset V $,
  $ \theta (S) \ge \lambda_2 (1 - s) $,
  where $ s = \frac{|S|}{|V|} $.
\end{theorem}
The proof makes use of the characteristic vector of S, 
$$ \chi_S (u) =
  \begin{cases}
    1 & \text{if $ u \in S $} \\
    0 & \text{otherwise.}
  \end{cases}
$$

\subsection{Common Graphs}

\begin{lemma}
  The Laplacian of $K_n$ (complete graph) has eigenvalue $0$ with multiplicity $1$ and $n$ with multiplicity $ n - 1 $. (check all vectors orthogonal to all-1s vector)
\end{lemma}

\begin{lemma}
  $ v, w $ are vertices of degree one that are both connected to another vertex $z$. The vector $\psi$ given by:
$$ \psi (u) = 
  \begin{cases}
    1 & u = v \\
    -1 & u = w \\
    0 & \text{otherwise.}
  \end{cases}
$$
is an eigenvector of the Laplacian of G of eigenvalue 1.
\end{lemma}

\begin{lemma}
The graph $ S_n $ (star graph) has eigenvalue $0$ with multiplicity $1$, eigenvalue $1$ with multiplicity $ n - 2 $ and eigenvalue $n$ with multiplicity $1$.
(use previous lemma, and trace of a matrix equal to both sum of diagonal entries and eigenvalues)
\end{lemma}

%%%%%%%%%%%%%%%%%%%%%%%%%%%%%%%%%%%%%%%%%%%%%%%%%%%%

\subsection{Graph Peoperties}
\paragraph{Conductance}
In graph theory the conductance of a graph G=(V,E) measures how "well-knit" the graph is: it controls how fast a random walk on G converges to a uniform distribution.
The conductance of a graph is often called the Cheeger constant of a graph as the analog of its counterpart in spectral geometry.

\paragraph{Small World Network}
A small-world network is a type of mathematical graph in which most nodes are not neighbors of one another, but most nodes can be reached from every other by a small
number of hops or steps. Specifically, a small-world network is defined to be a network where the typical distance $L$ between two randomly chosen nodes (the number of
steps required) grows proportionally to the logarithm of the number of nodes $N$ in the network.

$\displaystyle L\propto logN$

\subsection{Basic operations} 








